\section{Introduction}
\label{sec:intro}
\begin{figure}
    \begin{lstlisting}[style=Java-github]
public void removeAllAppenders() {
   if(appenderList != null) {
      int len = appenderList.size();
      for(int i = 0; i < len; i++) {
         Appender a = (Appender) appenderList.elementAt(i);
	 a.close();
      }
      appenderList.removeAllElements();
      appenderList = null;
    }
}
    \end{lstlisting}
    \caption{Data Race Situation Occurred in Log4j}
    \label{fig:code}
\end{figure}





Concurrent programming is widespread due to the presence of multi-core computer
system. However, developing an error-free and efficient program is difficult to
build due to the absence of proper thread synchronization. Improper
thread-synchronization often introduces concurrency faults more specifically
data-race. Data race is a situation when multiple threads will access a shared
variable and will try to access memory location at the same time.
Figure~\ref{fig:code} represents a code snippet of data race situation taken
from JACONTEBE benchmark \cite{}. In this example code, appenderList is a
shared data structure. At line 3, the size of the list is measured and if the
size is a non-zero positive integer then the values are extracted one by one.
At the end of this, the appenderList data structure is made null. The problem
may happen if multiple threads try to execute this code segment simultaneously
and one thread removes all the contents of the list, meanwhile another thread
is trying to read the value from the that data-structure.    

These data-race bugs are very difficult to expose because of the
non-deterministic thread-interleavings behavior. It is also common that race
condition often hides rare thread-interleavings during testing time but may
show up in the production phase. In the object-oriented language, thread-safety
contract is associated with each class and library. This kind of contract will
specify which methods could not be called concurrently. The thread-safety
violation occurs when a program violates this contract. At the end, it will
cause crashes or weird behaviors of the program.  Hence, researchers propose
few of the data race detection mechanism which can be categorized in two
groups; static race detection and dynamic race detection. Static data race
often consider flow, context and synchronization primitive information from a
program \cite{kahlon2009ESEC}, \cite{naik2006PLDI} whereas dynamic race
detection mechanism often uses run-time information of the code
\cite{pozniansky2007multirace}, \cite{li2019efficient}.   The main drawback of
static analysis is that it demands manual inspection of the source code and
there might have a large number of false bugs report. Hence, in this study we
implement thread safety violation detection mechanism for java program by
considering run-time information of the program. 

In our technique, we followed \cite{li2019efficient} paper as the foundation
paper. Here, we have four major steps such as byte code instrumentation,
interception point creation, setting trap mechanism and detect conflict pairs.
Byte code instrumentation is needed to get the runtime information of a program
by the stack modification. When we find our required API is found by the
analysis of jvm stack instructions, we create an interception point for that
API having object id, thread id, operation id, class name, line number and so
on. In the case of trap mechanism, we have few steps most importantly check for
trap, should delay, set for trap, delay and clear trap. Finally, at the end of
jvm shutdown, we record the number of conflicting pairs for different
interception point.

In the experiment, we considered three types of projects; toy projects
(developed by the authors of this report) \cite{toy_project_1},
\cite{toy_project_2}, JACONTEBE Benchmark \cite{lin2015jacontebe} and open
source Github projects \cite{githubProjects}. Results show that the project's
outcome is quite convincing. In the two toy projects, we saw that all the
conflicting pairs are reported by our algorithm. In the case of JACONTEBE, our
tool also produces satisfactory results by identifying conflicting pairs due to
the data race. In the case of open source github projects, we consider 21
projects which have multiple thread instance. For these datasets,  our technique
can potentially detect five conflicting data race pairs and we will report
these five to those projects to open new issues.


The rest of this paper is organized as follows. Section \ref{} presents an overview
of the existing works. Section \ref{sec:method} presents our algorithms to find thread safety violations. Section \ref{sec:experiment}
provides demonstration of the effectiveness of TSVD and lists the experimental
results to illustrate the performance of TSVD as compared to other existing
RV-Predict tool. Finally, Section \ref{} concludes the findings with future research directions.



%Here is the introduction text, where we describe our \Toolname{} along
%with our \TotalResults{} results.\Space{ The text in here is whited
%out in case we are over and need to cut text.}

\Comment{This is a comment, will not be rendered in PDF.}

%Citing our prior work~\cite{ShiETAL2014FSE, ShiETAL2019FSE, GyoriETAL2015ISSTA}.

%We found one \bug{} and it was awesome.
%We found two \bug{}s and it was even more awesome.
